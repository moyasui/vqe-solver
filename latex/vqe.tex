\documentclass[english,notitlepage,reprint,nofootinbib]{revtex4-1}  

\usepackage[utf8]{inputenc}


\usepackage{physics,amssymb}  % mathematical symbols (physics imports amsmath)
\include{amsmath}
\usepackage{graphicx}         % include graphics such as plots
\usepackage{xcolor}           % set colors
\usepackage{hyperref}         % automagic cross-referencing
\usepackage{listings}         % display code
\usepackage{subfigure}        % imports a lot of cool and useful figure commands
% \usepackage{float}
%\usepackage[section]{placeins}
\usepackage{algorithm}
\usepackage[noend]{algpseudocode}
\usepackage{subfigure}
\usepackage{tikz}
\usetikzlibrary{quantikz}
\hypersetup{
    colorlinks,
    linkcolor={red!50!black},
    citecolor={blue!50!black},
    urlcolor={blue!80!black}}


% ===========================================


\begin{document}

\title{Quantum Computing: Solving the Lipkin Hamiltonian Using Variational Quantum Eigensolver}
\author{Keran Chen} % self-explanatory
\date{\today}                             % self-explanatory
\noaffiliation                            % ignore this, but keep it.

%This is how we create an abstract section.
\begin{abstract}
    
Lorem ipsum dolor sit amet, consetetur sadipscing elitr, sed diam nonumy eirmod tempor invidunt ut labore et dolore magna aliquyam erat, sed diam voluptua. At vero eos et accusam et justo duo dolores et ea rebum. Stet clita kasd gubergren, no sea takimata sanctus est Lorem ipsum dolor sit amet.

\end{abstract}
\maketitle


% ===========================================
\section{Introduction}\label{sec:intro}
Quantum physics is all about solving the Hamiltonian. With the development of quantum computer, one would assume that it is only natraul to solve quantum problems with Hamiltonian.


%

% ===========================================
\section{Methods}\label{sec:methods}
%

% ===========================================
\subsection*{The algorithm}\label{sec:algo}
%

% ===========================================
\section{Results}\label{sec:results}
%
% ===========================================
\section{Discussion}\label{sec:discussion}
%

% ===========================================
\section{Conclusion}\label{sec:conclusion}

\onecolumngrid

%\bibliographystyle{apalike}
\bibliography{ref}


\end{document}
