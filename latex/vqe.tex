\documentclass[english,notitlepage,reprint,nofootinbib]{revtex4-1}  

\usepackage[utf8]{inputenc}


\usepackage{physics,amssymb}  % mathematical symbols (physics imports amsmath)
\include{amsmath}
\usepackage{graphicx}         % include graphics such as plots
\usepackage{xcolor}           % set colors
\usepackage{hyperref}         % automagic cross-referencing
\usepackage{listings}         % display code
\usepackage{subfigure}        % imports a lot of cool and useful figure commands
% \usepackage{float}
%\usepackage[section]{placeins}
\usepackage{algorithm}
\usepackage[noend]{algpseudocode}
\usepackage{subfigure}
\usepackage{tikz}
\usetikzlibrary{quantikz}
\hypersetup{
    colorlinks,
    linkcolor={red!50!black},
    citecolor={blue!50!black},
    urlcolor={blue!80!black}}


% ===========================================


\begin{document}

\title{Quantum Computing: Solving the Lipkin Hamiltonian Using Variational Quantum Eigensolver}
\author{Keran Chen} % self-explanatory
\date{\today}                             % self-explanatory
\noaffiliation                            % ignore this, but keep it.

%This is how we create an abstract section.
\begin{abstract}
    
Lorem ipsum dolor sit amet, consetetur sadipscing elitr, sed diam nonumy eirmod tempor invidunt ut labore et dolore magna aliquyam erat, sed diam voluptua. At vero eos et accusam et justo duo dolores et ea rebum. Stet clita kasd gubergren, no sea takimata sanctus est Lorem ipsum dolor sit amet.

\end{abstract}
\maketitle


% ===========================================
\section{Introduction}\label{sec:intro}
Quantum physics is all about solving the Hamiltonian. With the development of quantum computer, one would assume that it is only natraul to solve quantum problems with Hamiltonian.


%

% ===========================================
\section{Methods}\label{sec:methods}
%Lipkin Hamiltonian only moves particles between energy levels, not degenerate states.

% ===========================================
\subsection*{The algorithm}\label{sec:algo}
%


% ===========================================
\section{Results}\label{sec:results}
%
% ===========================================
\section{Discussion}\label{sec:discussion}
%

% ===========================================
\section{Conclusion}\label{sec:conclusion}

\section*{Appendix}\label{sec:app}

1. Consider the general 4x4 matrix M:

\[
M = \begin{bmatrix}
m_{11} & m_{12} & m_{13} & m_{14} \\
m_{21} & m_{22} & m_{23} & m_{24} \\
m_{31} & m_{32} & m_{33} & m_{34} \\
m_{41} & m_{42} & m_{43} & m_{44}
\end{bmatrix}
\]

2. Express the tensor products of I and $\sigma_z$:

\[
I \otimes I = \begin{bmatrix}
1 & 0 & 0 & 0 \\
0 & 1 & 0 & 0 \\
0 & 0 & 1 & 0 \\
0 & 0 & 0 & 1
\end{bmatrix}
\]

\[
I \otimes \sigma_z = \begin{bmatrix}
1 & 0 & 0 & 0 \\
0 & -1 & 0 & 0 \\
0 & 0 & 1 & 0 \\
0 & 0 & 0 & -1
\end{bmatrix}
\]

\[
\sigma_z \otimes I = \begin{bmatrix}
1 & 0 & 0 & 0 \\
0 & 1 & 0 & 0 \\
0 & 0 & -1 & 0 \\
0 & 0 & 0 & -1
\end{bmatrix}
\]

\[
\sigma_z \otimes \sigma_z = \begin{bmatrix}
1 & 0 & 0 & 0 \\
0 & -1 & 0 & 0 \\
0 & 0 & -1 & 0 \\
0 & 0 & 0 & 1
\end{bmatrix}
\]

3. Set up a system of equations by equating the corresponding elements of M and the tensor product combinations:

\[
\begin{aligned}
m_{11} &= a + b + c + d \\
m_{12} &= e + f + g + h \\
m_{13} &= i + j + k + l \\
m_{14} &= m + n + o + p \\
\ldots
\end{aligned}
\]

4. Solve the system of equations to find the values of the coefficients a, b, c, and d. This can be done analytically or using numerical methods such as matrix inversion or least squares.

By determining the specific values of the coefficients a, b, c, and d, you can rewrite any 4x4 matrix in the form:

\[
M = (a \cdot I \otimes I) + (b \cdot I \otimes \sigma_z) + (c \cdot \sigma_z \otimes I) + (d \cdot \sigma_z \otimes \sigma_z)
\]

Note that this representation is not unique, as different combinations of coefficients can represent the same matrix.
\onecolumngrid

%\bibliographystyle{apalike}
\bibliography{ref}


\end{document}
